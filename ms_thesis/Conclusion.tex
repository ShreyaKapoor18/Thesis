\documentclass[msthesis.tex]{subfiles}

\begin{document}
\chapter{Conclusion}
\iffalse
Interpretability-accuracy tradeoff. 
With the solver - 42 features (10 nodes)
with the baseline - much more number of nodes for the same features.
\fi
Analyzing the properties of structural brain networks is a challenging task. Especially classification based on dense graphs derived from brain connectivity requires a well assessed scientific methodology. The pipeline implemented in this work is tailored towards accurate classification from structural brain connectivity. Three major characteristics of the pipeline make it apt for usage in clinical as well as behavioral research. These aspects are the consideration of graph topology, visualization at each step and a generalized approach. 

First, the graph topology incorporated using the \gls{MEWIS} technique provides interpretable and visualizable results. This framework enables effective classifier training and statistical analysis. The structure of the output subgraph can be controlled by specifying a number of parameters: the number of nodes to be preserved, the type of ranking, and the type of connectivity measure. The extraction of a k-induced subgraph gives the advantage of preserving edge information along with node information over the \gls{GMWCS}. Furthermore, the number of edges can also be controlled indirectly by specifying the number of nodes. This makes it a better alternative to the state-of-the art filter based methods used in neuroimaging studies. 

Second, all the steps right from the preprocessing of the connectome to the features selected for model training have a visual analog. Detailed visualizations that help evaluate biological correspondence of the data preparation and selection have been incorporated into the pipeline. These visualizations aid the evaluation of significant biases that can influence the exactness of an \textit{in-silico} model of brain connectivity. 

Third, the pipeline is a generalized method for classification using structural brain connectivity. It can be implemented for continuous or categorical variables. Special consideration has been given to remove the effects of information loss for continuous variables. It is also classifier independent and computationally efficient. An extensive analysis can be carried out by combining the \gls{MEWIS} feature selection technique with a wide range of classifiers.
\iffalse
Linear coefficient was used to account for the linear relationship between a particular feature and the continuous variable. For a binary variable group based differences were evaluated by grouping subjects. In this way information supervised filter methods were combined with graph topology considerations.
\fi

Overall, the method is novel but is not comprehensive on addressing all the challenges that come in the way of modeling structural brain connectivity. It gives a controllable trade-off between model performance and interpretability. The pipeline could further be improved by encoding more \textit{a priori} knowledge from neurobiology and consideration of a larger dataset and using different imaging modalities. Another important observation could be to see if the features selected by the diffusion matrices correspond to the features selection by using functional connectivity matrices formed using fMRI data.
\iffalse
It is a good stepping stone on trying to model the brain and guide clinical decisions. It achieves good performance on gender classification/ Gender differences important guide decision making in Alzheimer's which is more common in females than in males.
\fi
\end{document}

