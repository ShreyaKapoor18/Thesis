\documentclass[msthesis.tex]{subfiles}

\begin{document}
\chapter{Conclusion}
\iffalse
Interpretability-accuracy tradeoff. 
With the solver - 42 features (10 nodes)
with the baseline - much more number of nodes for the same features.
\fi
Analyzing the properties of structural brain networks is a challenging task. Especially classification based on dense graphs derived from brain connectivity requires a well assessed scientific methodology. The pipeline implemented in this work is a is tailored towards accurate classification from structural brain connectivity. Three major characteristics of the pipeline make it apt for usage in clinical as well as behavioral research. These aspects are the consideration of graph topology, visualization at each step and a generalized approach. 

First, the graph topology incorporated using the MEWS technique provides interpretable and visualizable results. With this framework, it enables effective classifier training and statistical analysis. The structure of the output subgraph can be controlled with the number of nodes to be preserved, the type of ranking or the type of connectivity measure to use. Especially, the number of nodes to be preserved was implemented by me can be used to rank node importance additional to feature importance. The number of edges can also be controlled indirectly with the number of nodes. This makes it a better alternative to the state of the art filter based methods used in Neuroimaging studies. 

Second, all the steps right from the preprocessing of the connectome to the features selected for model training had a visual analog. Detailed visualizations that help evaluate biological correspondence of the data preparation/selection. This type of visualization can help evaluate significant biases/errors that come into the picture while making an \textit{in-silico} model of brain connectivity. DTI images are of anatomical correspondence and any features which are derived on their basis need to be evaluated visually for neurobiological correspondence. 

Third, the pipeline can be implemented for any type of target variable. Be it a categorical/binary variable such as gender or a continuous variable such as a personality trait. One important thing to note is that special consideration was given to remove the effects of information loss for categorical variables. Linear coefficient was used to account for the linear relationship between a particular feature and the continuous variable. For a binary variable group based differences were evaluated by clubbing subjects in different groups. 

Overall, the method is novel but is not comprehensive on addressing all the challenges that come in the way of modeling structural brain connectivity. It also gives a controllable trade-off between model performance and interpretability. The pipeline could further be improved by encoding more \textit{a priori} knowledge from neurobiology and consideration of a larger dataset, using different modalities and seeing if the features selected by the diffusion matrices correspond to the features selection by using functional connectivity matrices formed using fMRI data.

\iffalse
It is a good stepping stone on trying to model the brain and guide clinical decisions. It achieves good performance on gender classification/ Gender differences important guide decision making in Alzheimer's which is more common in females than in males.
\fi
\end{document}

