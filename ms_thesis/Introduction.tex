\documentclass[msthesis.tex]{subfiles}

\begin{document}
\chapter{Introduction}
One of the greatest scientific challenges in the $21^{st}$ century is to understand the structure (composition) and function of the human brain. How do intricate network of cells (neurons) organize themselves at different scales and give rise to behaviour, emotion and intelligence has remained an open question in neuroscience since ages. In recent years, there has been a surge of interest in understanding the human brain in health and disease, and on a more philosophical level, trying to understand what makes us, us? 

Research in Neuroscience has greatly benefited from advances in Neuroimaging data acquisition, analysis and retrieval. The decreasing cost of data storage, computational memory resources has lead to development of imaging standards and made wide range studies possible. Increasing efficiency and accuracy of non-invasive imaging modalities such as Magnetic Resonance Imaging (MRI), CT and PET have contributed significantly towards the human brain in health and disease. Among these, MRI is one of the most preferred modalities due to the diversity of information that can be generated using one scanner only. Each of the four types structural MRI, Task-activated functional MRI, Diffusion MRI and Resting state fMRI can be used to carry out different types of research (\cite{van2016human}). 

Excitement from investigators has brought about the progress of ambitious projects such as the Human Connectome Project launched in 2009 (\cite{van2016human}). It explores the research area known as 'Connectomics' (previously known as hodology), the study of brain's structural and functional networks.  It aims at creating a map of the brain divided into functionally distinct areas known as 'parcels', understand how these areas are connected to each other and how they contribute to behaviour. This framework is also important to understand what goes wrong during Neurological disorders \cite{sala2015reorganization}. 

Structural brain connectivity is the study of white matter tracts in the brain and can be measured using Diffusion Tensor Imaging (DTI). Functional connectivity is based on information exchange between different brain regions and can be measured using functional Magnetic Resonance Imaging (fMRI). Each of these connectivities have their own importances but DTI based structural connectivity is of special importance in medical applications as it provides insight into anatomical connections in the brain of individual patients, can aid precision medicine (\cite{cociu2017multimodal}).

Machine Learning has become an indispensable tool in Neuroscience/for medical research due to the fact that the number of subjects in the study is often much lesser than the number of features in the data. It has often been used to make predictions based on MRI scans and classify diseased states of patients. However, for the analysis of brain connectivity there is a need for the ML algorithm to understand the topological properties of brain organization. It is also important to realize that the brain connectivity can be seen as a dense graph and there are not too many graph based machine learning classifiers. Further, even if these classifiers are able to work with graphs, they suffer from a lack of interpretability. 

\iffalse

through neurotransmitters and ... action potentials. (explain how different parts of the brain are connected and what types of chemical reactions lead to behavior, intelligence, disease etc.) In terms of physics, we can consider the brain to be a dynamic system in which an ensemble of cells communicate with each other for (maintaining proper function)? Number of connections in the brain, number of neurons
The brains ability to learn arises from modification, strengthening and pruning of large number of connections between its neurons. When a child grows there are x number of neurons, as we \cite{10.3389/fnins.2018.00525}
\fi

\cite{10.3389/fnagi.2017.00329}

\end{document}
