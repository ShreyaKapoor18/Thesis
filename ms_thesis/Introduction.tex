\documentclass[msthesis.tex]{subfiles}

\begin{document}
\chapter{Introduction}
One of the greatest scientific challenges in the $21^{st}$ century is to understand the structure and function of the human brain. The mechanism in which intricate networks of cells organize themselves at different scales and give rise to behavior, emotion, and intelligence has remained a topic of discussion in neuroscience for ages. In recent years, there has been a surge of interest to understand the human brain in health and disease. Technological advances are taking humanity closer to answering the question: what makes us, us? 

Research in Neuroscience has greatly benefited from advances in neuroimaging data acquisition, analysis, and retrieval. The decreasing cost of data storage and computational memory resources has lead to the development of imaging standards and made a wide range of studies possible. Increasing efficiency and accuracy of non-invasive imaging modalities such as \gls{MRI}, \gls{CT}, and \gls{PET} have contributed significantly towards understanding the human brain in health and disease. Among these, \gls{MRI} is a popular modality due to the diversity of information that can be generated using a single scanner. Each of the four types of structural MRI, task-activated functional \gls{MRI}, diffusion \gls{MRI}, and resting-state \gls{fMRI} can be used to carry out different types of research \citep{van2016human}. 

The excitement from investigators has brought about the progress of ambitious projects such as the \gls{HCP} launched in 2009 \citep{van2016human}. The \gls{HCP} explores the research area known as \textit{connectomics} (previously known as \textit{hodology}, the study of the brain's structural and functional networks. Connectomics aims at creating a map of the brain that is divided into functionally and structurally distinct areas known as parcels. Understanding the contribution of connections between regions is crucial to analyze the emergent properties of brain structure and function. This framework is also important to understand defects caused due to neurological disorders as well as emotion and cognition \citep{sala2015reorganization}. 

Structural brain connectivity is the study of white matter tracts in the brain and can be measured using \gls{DTI}. Functional connectivity is based on information exchange between different brain regions and can be measured using \gls{fMRI}. Each of these connectivities has its importance. However, \gls{DTI} based structural connectivity is of particular importance for medical applications as it provides insight into anatomical connections within an individual's brain, and can aid precision medicine \citep{cociu2017multimodal}.

Machine Learning has become an indispensable tool in neuroimaging due to the  \textit{curse of dimensionality} in such studies. Here, the number of subjects in the study is often much lesser than the number of features. It is often used to make predictions based on \gls{MRI} scans and to classify diseased states of patients. However, for the analysis of brain connectivity there is a need for the machine learning algorithm to understand the topological properties of brain organization. Generic classifiers such as Naive Bayes are not well suited for brain networks due to the density and computational complexity of such networks. Traditional classifiers suffer from a lack of interpretability and often do not lead to results with neurobiological correspondence \citep{10.3389/fnagi.2017.00329}. Hence, it is required to incorporate the advantages of graph theoretic methods and machine learning algorithms for making neurobiologically and statistically significant predictions. 



\end{document}
