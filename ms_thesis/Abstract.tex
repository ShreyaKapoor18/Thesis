\documentclass[msthesis.tex]{subfiles}

\begin{document}
\chapter{Abstract}
\thispagestyle{empty}

Advances in non-invasive Neuroimaging modalities has lead to a growing interest in investigating neural underpinnings of human cognition and neurological disorders. Analysis of human brain connectivity or the human connectome has hence become an active area of research. On one hand, Computational methods, especially the ones based on graph theory have played a significant role in understanding the topological organisation of brain networks. On the other hand, machine learning classifiers are now being used in Neuroimaging studies to classify diseased states in patients. In this work, we propose a method to combine the utilities of graph theoretic methods with Machine Learning classifiers in order to obtain discriminative subnetworks in the brain, which make the classification results more interpretable. The novel component of this thesis is to reduce the graphical network derived from structural brain connectivity by approximating the solution to the Maximum Edge Weight Subgraph (MEWS) problem that falls into the class of NP-hard problems. The comprehensive analysis pipeline includes the comparison of feature selection based techniques based on statistical analysis, and the MEWS based method. Experimental results from this technique show (to be filled after discussion)

\end{document}
