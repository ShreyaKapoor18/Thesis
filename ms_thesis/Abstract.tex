\documentclass[msthesis.tex]{subfiles}

\begin{document}
\chapter{Abstract}
\thispagestyle{empty}

Advances in non-invasive Neuroimaging modalities has lead to a growing interest in investigating the underpinnings of human cognition and neurological disorders. Analysis of human brain connectivity or the human 'connectome' has hence become an active area of research. Computational methods based on graph theory have played a significant role in understanding the topological organisation of brain networks. Simultaneously, machine learning algorithms are now being used in Neuroimaging studies to classify diseased states in patients. In this work, we propose a method to combine the utilities of graph theoretic methods with Machine Learning classifiers in order to obtain discriminative subnetworks in the brain. It is done to make the classification results more interpretable. The novel component of this thesis is to reduce the graphical network derived from structural brain connectivity by approximating the solution to the Maximum Edge Weight Subgraph (MEWS) problem that falls into the class of NP-hard problems. Firstly, the experimental results from show that MEWS subgraphs can achieve about 94\% area under roc curve for gender classification. Secondly, the consideration of graph topology makes the results from classification more interpretable.

\end{document}
