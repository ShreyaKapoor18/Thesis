\documentclass[msthesis.tex]{subfiles}

\begin{document}
\chapter{Abstract}
\thispagestyle{empty}

Advances in non-invasive Neuroimaging modalities have led to a growing interest in investigating the underpinnings of human cognition and neurological disorders. Analysis of human brain connectivity or the human connectome has hence become an active area of research. Computational methods based on graph theory have played a significant role in understanding the topological organisation of brain networks. Simultaneously, machine learning algorithms are now being used in Neuroimaging studies to classify diseased states in patients. In this work, we propose a method to combine the utilities of graph theoretic methods with Machine Learning classifiers in order to obtain discriminative subnetworks in the brain. The novel component of this thesis is to reduce the graphical network derived from structural brain connectivity. Subnetworks are obtained by approximating the solution to the Maximum Edge Weight Subgraph (MEWS) problem that falls into the class of NP-hard problems. This feature selection technique makes the classification results more interpretable as compared to traditional filter based methods. The experimental results from our analysis show that MEWS subgraphs can achieve 95\% area under ROC curve for gender classification. Furthermore, it is a generalized method which is independent of the classification task and hence can be useful for discovery of disease biomarkers.

\end{document}
