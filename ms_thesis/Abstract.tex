\documentclass[msthesis.tex]{subfiles}

\begin{document}
\chapter{Abstract}
\thispagestyle{empty}

Advances in non-invasive neuroimaging modalities have led to a growing interest in investigating the underpinnings of human cognition and neurological disorders. Analysis of human brain connectivity or the human connectome has hence become an active area of research. Computational methods based on graph theory have played a significant role in understanding the topological organization of brain networks. Simultaneously, machine learning algorithms are now being used in neuroimaging studies to classify diseased states in patients. In this work, we propose a method to combine the utilities of graph theoretic methods with machine learning classifiers to obtain discriminative subnetworks of the brain. The novel component of this thesis is to reduce the graphical network derived from structural brain connectivity. Subnetworks were extracted by approximating the solution to the \gls{MEWIS} problem that falls into the class of NP-hard problems. The use of \gls{MEWIS} as a feature selection technique makes our classification results more interpretable as compared to traditional filter based methods. The experimental results from our analysis show that MEWS subgraphs can achieve 95\% \gls{AUC} for gender classification. Furthermore, it is a generalized method that is independent of the classification task and can hence be useful for the discovery of disease biomarkers.

\end{document}
