\documentclass[msthesis.tex]{subfiles}

\begin{document}
\chapter{Discussion}


\section{Connectivity features}
There were three types of features used as representatives of structural brain connectivity. The number of streamlines, the mean FA of streamlines and the mean streamline length. Generally, the number of streamlines seemed to have performed much better than the other two features considering the area under the curve for classification of gender on an independent test set. It erformed better in cases of using the baseline and the solver based approach.

The greater performance of the number of streamlines using the solver based approach seems plausible due to the incoporation of a particular constraint in the pipeline. A particular feature in the connectome is included in the input graph for the solver if and only if there is atleast one streamline present for all subject. Such a condition remained unique to this feature. Perhaps trying such conceptual \textit{a priori} knowledge with the other two features could yield different results. The number of streamlines were not performing better due to the standardization. 

\section{Statistical Coefficient}
There were numerous possibilities for the edge weights in the input graph for the solver. It was maintained that the input graph edge weights shall represent a statistical coefficient which represents the strength of the connections between the two nodes/brain regions determined by the cortical parcellation. The coefficients chosen were the f-score, the pearson correlation coefficient and the p-value of the t-test. 

The pearson correlation coefficient in fact did work because it considers the linear nature of the personality trait coefficients. This type of feature selection is well reported in literature for Neuroimaging data considering continuous variables. It performed well for the baseline experiments as well as the solver based feature selection. For both the cases the absolute value of the pearson correlation coefficient was taken because only the correlation was important, whether it is positive or negative correlation was not a matter of concern for the analysis.

The f-score and the t-test were based on the binarization of the target variables according to the median values of the feature from the training set, this might lead to information loss and hence their lower numerical values. 

\section{Node weights}
Different types of node weightings were tried to form the subgraph besides from the values presented in the results. The nodes were given sets of random weights, average of all incoming edge weights, maximum of all incoming edge weights and constant values in different numerical ranges. 

In cases where the node weights were positive the cost on preserving the nodes was 
\section{Deciding on the number of nodes}
The number of nodes to be given to the solver had to be determined using experimentation. The soolver only starts preserving edges from three nodes. Which is actually quite significant since starting with three nodes, we actually start to obtain a connected graph. Also the number of edges as a function of the number of nodes follows between a linear and a quadratic graph which is quite relevant according to mathematical formulation. 

Independent of the type of feature used the number of edges obtained as a function of the number of nodes remained almost the same. Further, it was seen that smaller subgraphs were more effective than larger subgraphs. The classification metrics remain the most important determinants of deciding the number of nodes. 

\section{Classification with Personality traits}
The classification of personality traits was a methodologically challenging task. The personality traits are continuous in nature, binning them for a classification task definitely lead to information loss. However, this information loss was incorporated in the classification technique by determining the pearson correlation coefficient of the personality traits with the feature.The evidence of information loss was also observed by trying to give the input graphs based on t-test and fscores to the solver for classification of personality traits. However, t-test coefficients and the fscore coefficients had low standard deviations which made account for the fact that they could not be able to well distinguish between different feature relations to personality traits. 
\section{Classifier feature Combination}
Classifier MLP with Num streamlines for Gender classification
Classifier SVM with mean FA for personality traits


\end{document}
