\documentclass[msthesis.tex]{subfiles}

\begin{document}
\chapter{Discussion}
In this work classification based on the graph was the major challenge. A large combination of possibilities for designing the classification task was tried in order to obtain the most interpretable and
statistically significant results. The goal was to not only gain the maximum classification accuracy but to also make a generalized framework in which each step can be visualized. The architecture remains specific
to brain connectivity but generalizable enough to predict different target variables. Major considerations for the task design such as connectivity features and graph structure are presented in the subsequent sections.

\section{Connectivity features}
There were three types of features used as representatives of structural brain connectivity. The number of streamlines, the mean FA of streamlines and the mean streamline length. Generally, the number of streamlines seemed to have performed much better than the other two features considering the area under ROC curve for classification of gender on an independent test set. It performed better with both baseline and solver based methods, regardless of the labels i.e. personality trait or gender classification. The greater performance of the number of streamlines using the solver based approach seems plausible due to the incorporation of a particular constraint in the pipeline. This constraint was that a feature in the connectome is included in the input graph for the solver if and only if there is at least one streamline present for all subject. Such a condition remained unique to number of streamlines. Perhaps trying such conceptual \textit{a priori} knowledge with the mean FA and mean streamline length could yield different results. 

Further, streamline quantification bias in the analysis pipeline was corrected with the help of informed fiber filtering. This could have made the number of streamlines an effective measure of connectivity. There is currently no ground truth regarding which of these three connectivity measures is the best to formulate the relationship between any two ROIs. The mean FA could also provide a biologically accurate meaning of connectivity considering it is independent of size of a person's brain unlike the mean streamline length. The mean streamline length could not have performed well because it might be sensitive to the differences in the sizes of subject's brains. The brain size could then become a confounding factor where higher streamline lengths are an artefact of bigger brain size.

\section{Statistical Coefficient}
There were numerous possibilities for the edge weights in the input graph for the solver. It was maintained that the input graph edge weights shall represent a statistical coefficient which represents the strength of the connections between any two brain regions determined by the cortical parcellation. The coefficients chosen were the f-score, the pearson correlation coefficient and the p-value of the t-test. 

The pearson correlation coefficient worked well as a filter method for the personality traits. This was because it considers the linear nature of the personality trait coefficients. This type of feature selection is well reported in literature for Neuroimaging data considering continuous variables. It performed well for the baseline experiments as well as the solver based feature selection. For both the cases the absolute value of the pearson correlation coefficient was taken because only the correlation was important, whether it is positive or negative correlation was not a matter of concern for the analysis. The f-score and the t-test were based on the binarization of the target variables according to the median values of the personality trait from the training set, this might lead to information loss and hence their lower numerical values. 

For gender, which is already categorical, the t-test and the fscores fitted as a good scoring technique. There was not any information loss considering these two statistics measure similar attributes of a particular feature. They measure how well does a particular feature separate the two classses. This type of selection worked quite well with gender variables.

\section{Node weights}
Different types of node weightings were tried to form the subgraph besides from the values presented in the results. The nodes were given sets of random weights, average of all incoming edge weights, maximum of all incoming edge weights and constant values in different numerical ranges. 
In cases where the node weights were positive the cost on preserving the nodes was 
\section{Deciding on the number of nodes}
The number of nodes to be given to the solver had to be determined using experimentation. The soolver only starts preserving edges from three nodes. Which is actually quite significant since starting with three nodes, we actually start to obtain a connected graph. Also the number of edges as a function of the number of nodes follows between a linear and a quadratic graph which is quite relevant according to mathematical formulation. 

Independent of the type of feature used the number of edges obtained as a function of the number of nodes remained almost the same. Further, it was seen that smaller subgraphs were more effective than larger subgraphs. The classification metrics remain the most important determinants of deciding the number of nodes. 

\section{Classification with Personality traits}
The classification of personality traits was a methodologically challenging task. The personality traits are continuous in nature, binning them for a classification task definitely lead to information loss. However, this information loss was incorporated in the classification technique by determining the pearson correlation coefficient of the personality traits with the feature.The evidence of information loss was also observed by trying to give the input graphs based on t-test and fscores to the solver for classification of personality traits. However, t-test coefficients and the fscore coefficients had low standard deviations which made account for the fact that they could not be able to well distinguish between different feature relations to personality traits. 



\end{document}
